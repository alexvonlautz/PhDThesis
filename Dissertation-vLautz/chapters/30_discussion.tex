% !TEX root = ../Dissertation-vonLautz.tex

\chapter{General Discussion}

%%%%%%%%%%%%%%%%%%%%%%%%%%%%%%%%%%%%%%%%%%%%%%%%%%
In the research summarized in the previous section, we gained new insights into the role of neural oscillations ascribed to cognitive functions, especially working memory modulated beta and gamma, decision-related beta power and the supramodal nature of the CPP for tracking decisions. We were able to uncover suspected, but previously unknown modulations of prefrontal gamma and parietal beta power by vibrotactile frequencies held in working memory. We showed that the functional roles attributed to beta oscillations in the tactile task hold up in a visual variant, indicating a general, supramodal mechanism. Similarly, we demonstrated that the CPP tracked the evolution of a decision variable during a comparison task that was informed by working memory. Finally, we made a first step at relating the CPP to choice-related beta power and provide evidence that late parietal signals might reflect confidence.

%\section{Unifying accounts of prefrontal beta band oscillations during working memory}

%All experiments reviewed here replicated the finding that beta band power from IFG is parametrically modulated by abstract quantitative information (vibrotactile frequency or coherence) held in working memory during a short interval between two stimuli. \textcite{Spitzer2010} had identified this effect at 20-25 Hz with EEG during a similar vibrotactile SFC task. Interestingly, when we used a visual variant of the task, but the same EEG equipment as these authors, the effect was surprisingly similar at 18-26 Hz, however, was found significantly higher (30-35 Hz) in MEG recordings on almost the same vibrotactile task. Because in EEG the skull acts as a low-pass filter \parencite{Pfurtscheller1975} and the observed effect may be an epiphenomenon of averaging further temporally smeared beta bursts, it is quite possible that the prefrontal oscillations even extend into the gamma range. Recent monkey recordings appear to concur with a representation in higher beta oscillations, with a medial premotor beta peak above 25 Hz \parencite{Haegens2017}. This may particularly important, as lower beta modulations (13-20 Hz) occur jointly with alpha \parencite{Hanslmayr2009} and decrease in task-relevant areas while higher beta-band rhythms (20-35) mirror gamma and increase with engagement \parencite{Tallon-Baudry1998}. It may therefore be necessary, in the future, to separate lower beta activity more clearly from higher beta oscillations to avoid grouping them in one beta band.
%Our findings in the prefrontal cortex were clearly in the upper beta band and appeared to be related to active processing rather than inhibitory in nature. Enhanced beta has been hypothesized to signal the “status quo” \parencite{Engel2010} of maintaining the current sensorimotor or cognitive state. However, our findings go beyond overall changes. We provide evidence that beta oscillations hold information about the content (vibrotactile frequency or RDM coherence) of working memory on a given trial, joining a growing body of evidence for such a role \parencite{Spitzer2010,Spitzer2011,Wimmer2016}. This type of feature specific activity has also been found in neuronal spiking and high frequency LFPs \parencite{Pesaran2002,Romo1999} and has been studied extensively with fMRI \parencite{Christophel2012,Christophel2017,Uluc2018,Wu2018}. Yet, the relationship of EEG beta band oscillations with the BOLD contrast as well as monkey electrophysiology remains unclear, while the gamma band might be more closely associated with these other neural measurements \parencite{Michels2010, Whittingstall2009}. 

\section{Diverging findings with MEG and EEG during sequential comparison tasks}

One major difference between our MEG recordings and previous EEG studies was that we observed parametric changes in gamma spectral power by the to-be-maintained frequency $f1$. One explanation why parametric WM in high frequency gamma oscillations was not detected in the large amount of previous vibrotactile EEG studies \parencite{Herding2016,Spitzer2010,Spitzer2014,Spitzer2011,Spitzer2012} is the previously mentioned nature of the human skull to act as a low-pass filter \parencite{Pfurtscheller1975}. Specifically, EEG and MEG can exhibit distinct frequency versus power relationships in high frequencies, because the capacitive properties of the extracellular medium, i.e. skin and scalp muscle artefacts, distort the EEG, but not the MEG signal \parencite{Buzsaki2012,Dehghani2010,Demanuele2007}. In addition, we used a multitaper approach based on Slepian sequences with a fixed window of 200 ms in the MEG study compared to a window of 400 ms in previous EEG recordings. Because of these differences, we also used the shorter window for exploratory analysis in more recent experiments, for example study 2, resulting in the same negative gamma modulation by the abstract quantity held in working memory as observed with MEG. For prospective WM studies or an eventual meta-analysis of the present findings I therefore also recommend using a short window for multitaper analysis of higher frequencies.
In addition to gamma, the MEG recordings differed from EEG recordings in one more area: the IPS. We found that low beta band power (10-20) from this area was parametrically modulated by the abstract quantity retained in WM. There are two methodological reasons that could account for why we detected this modulation with MEG and not EEG \parencite[cf.][]{Spitzer2014}. One, MEG has a higher signal-to-noise ratio for shallow sources \parencite{Goldenholz2009}. Two, MEG is more sensitive to sulcal than gyral sources, because it is blind to radial dipoles, biasing source analysis in favor of sulcal sources \parencite{Ahlfors2010}. However, while unexpected from the previous EEG literature, the involvement of the IPS in quantity processing was not wholly surprising. Concurrent to our research, fMRI studies found a role of the IPS for vibrotactile, visual, and auditory frequency maintenance \parencite{Uluc2018,Wu2018}. Moreover, similar to the abstract stimuli we employ, tasks using concrete numbers have found a direct link between BOLD-responses in the IPS and quantity with MVPA for fMRI \parencite{Eger2009}. This finding builds upon a body of work with nonhuman primates that has revealed a crucial involvement of intraparietal regions for the encoding of quantitative features that are ordered along a continuum \parencite{Jacob2012,Nieder2016}, including supramodal frequency \parencite{Vergara2016}. Furthermore, the IPS has been well-established as a hub for working memory in junction with the prefrontal cortex in studies on capacity limits and appears to be essential for short term object retention \parencite{Todd2004,Todd2005,Vogel2004,Xu2006}. Therefore, a role of the IPS in conjunction with the PFC has been well-established, yet it remains unclear what role beta band oscillations at low frequencies contribute to working memory in this area. In particular, the low beta we observed with MEG includes frequencies associated with the mu rhythm \parencite{Chatrian1959,Gastaut1954}, whose functional role has been viewed as alpha-like suppression for somatosensation \footnote{Note that there is a different view on mu as a correlate of the mirror neuron system: \footcite{Naeem2012,Pineda2005}}. Contrary to this interpretation, we did not observe overall ERD/ERS, but a parametric modulation of lower beta/mu power by the content held in working memory. It is therefore unlikely that the observed effect reflects a generic inhibition or gating mechanism. One possibility is that the inhibition is content specific, explicitly because another stimulus is being presented subsequently that is sure to be different in vibrotactile frequency. Thus, the stimulus frequency held in working memory could be inhibited. However, the parametric modulation of IPS did not extend to the time of f2 stimulation, rendering this interpretation unlikely. Similarly, the mu rhythm is associated with attention \parencite{Anderson2011} and it is possible that participants paid more attention to higher frequencies. This is unlikely for two reasons. First, we did not observe behavioral effects in this direction, and second, prefrontal gamma band and occipital alpha would be expected to be similarly modulated, which we also did not observe (in this direction). Therefore, since a role of the IPS for numerosity processing and working memory is well-established and it appears that our observations are difficult to reconcile with the inhibitory nature of low frequencies, I speculate that our observations reflect an active maintenance mechanism, possibly interacting with prefrontal gamma.
\section{A frontoparietal beta-gamma code}
One idea could be that the IPS maintains the information and is top-down controlled by prefrontal areas in a periodic replay based upon interactions of beta and gamma. This idea is in analogy to computational modeling of neuronal firing patterns in animals proposing that working memory arises from periodically reactivating the content held in working memory, guided by gamma and theta oscillations \parencite{Fuentemilla2010,Jensen2005,Lisman1999,Lisman1995}. This concept likely extends beyond the theta-related hippocampus to other areas and frequency bands \parencite{Lundqvist2016,Mongillo2008} and could be modulated on short time scales by attention \parencite{Awh1998}. More so, there is evidence of PFC-PPC coupling in the beta and delta bands, with delta reflecting task-irrelevant stimulus dimensions and beta only those immediately relevant \parencite{Antzoulatos2016}. Yet so far, a gamma-beta relationship has only been shown within the prefrontal cortex \parencite{Lundqvist2016} and not across frontoparietal areas. 
However, this same idea may serve to explain the pattern of concurrent high beta band increases and gamma decreases with the abstract quantity held in working memory I observed both with MEG and EEG in separate tactile and visual studies. \textcite{Lundqvist2016} observed brief gamma (45-100 Hz) and beta (20-35 Hz) bursts during single cell and LFP recordings of monkeys performing a working memory task. The gamma bursts increased during encoding and recall, while the beta bursts reflected a default network state that was interrupted by gamma. My findings could be an epiphenomenon of such a coding scheme, but reflected in mean power differences, because we average over many trials and smooth out individual bursts. 
Under close examination this interpretation speaks against working memory as a function grounded in sustained prefrontal firing rates \parencite{Funahashi1989,Fuster1971,Goldman-Rakic1995,Pasternak2005}. Studies demonstrating continuous delay activity have relied heavily on trial and spike averaging, convoluting more complex single trial dynamics \parencite{Rainer2002,Shafi2007}. Indeed, similar to the previously mentioned beta-gamma patterns \parencite{Lundqvist2016}, most neurons are variable in their spiking behavior in both timing and duration throughout retention intervals and show dynamic coding schemes transitioning between coding states \parencite{Cromer2010,Durstewitz2006,Spaak2017,Stokes2013}. More so, recent neuroimaging studies suggest that working memory can be ‘activity silent’ when stimuli are unattended or irrelevant for current task demands \parencite{Lewis-Peacock2012,Stokes2015,Wolff2015,Wolff2017}. While the role of such silent states is currently under high-level debate \parencite{Christophel2018}, evidence converges that working memory does not rely on sustained prefrontal firing as a solitary mechanism \parencite{Lundqvist2018,Spaak2017}. My data agrees with this notion. First, contrary to \textcite{Haegens2010} we did not observe an overall increase of gamma power during working memory as would be expected from sustained firing, while replicating findings in alpha and beta. Second, irrespective of whether gamma power changes reflected bursting, the observed changes in gamma were in a finite time window and not sustained throughout the whole interval. Third, while single-trial dynamics remain unclear, the pattern of gamma decrease with concurrent beta increase in PFC and PPC hint at a relationship between these frequency bands for working memory. 
The key to a unifying explanation for these effects may be provided by a very recent monkey study: \textcite{Lundqvist2018} found that gamma increased, and beta decreased shortly before items in working memory had to be used for decision making, while gamma decreased and beta increased when stimuli were not needed anymore. The authors interpret this as beta oscillations regulating control over gamma and working memory, a view summarily fitting to our results and recent investigations into the role of beta “beyond the status quo” \parencite{Haegens2017,Ludwig2018,Lundqvist2018,Spitzer2017}. 
In this view, beta oscillations provide a mechanism to guide neural ensembles for the (re-)activation of maintained information. This builds on the observation that beta facilitates top-down driven communication across long distances and cortical areas \parencite{Antzoulatos2016,Arnal2012,Bastos2015,Engel2010,Michalareas2016,Sejnowski2006,Siegel2012,Varela2001,Wang2010}, but beyond static maintenance can be characterized as a dynamic mechanism that can facilitate content-specific encoding and read-out by “waking up” in the form of short temporal bursts \parencite{Fries2015,Jones2016,Lundqvist2018,Spitzer2017}. The question remains however, whether beta facilitates information “wake up” over long range connections, e.g. from sensory areas or if it is a mechanism of central processing in the prefrontal cortex. This is particularly interesting, because I have found consistent parametric modulations of beta oscillations in the PFC across visual and tactile tasks while none from sensory areas.
\section{Distributed codes or central working memory?}
While we observed working memory related activity consistently only in the prefrontal cortex, recent accounts also focus on a role for parietal and sensory cortices \parencite{Bettencourt2015,Christophel2017,Sreenivasan2014,Xu2015}. In particular, studies using MVPA on fMRI recordings during the maintenance of precise visual details could reliably decode stimulus content from sensory cortices, yet failed in frontal areas \parencite{Christophel2012,Emrich2013,Riggall2012}. When operationalizing the retention of vibrotactile frequencies however, fMRI has revealed multivariate parametric codes in prefrontal and sensory areas \parencite{Schmidt2017,Wu2018}. Thus, multiple avenues of research into working memory have found very distinct regions to be involved - how can these findings be consolidated?
One idea is that the locus of working memory follows the processing in terms of the cortical hierarchy \parencite{Eriksson2015,Fuster2012,Zimmer2008}. More so, \textcite{Christophel2017} postulate that for one, all cortical regions can maintain information over a short period of time and two, the nature of the task dictates the relevant region in the cortical hierarchy depending on low level sensory features and the level of abstraction of the to-be-remembered stimulus. With such an interpretation of previous results, the areas involved in working memory can range from the prefrontal cortex for abstract, complex stimuli to low-level features in primary sensory cortices. Contrary to early MVPA fMRI studies tasking volunteers to remember low-level sensory details, we used abstract magnitude information, either in the form of vibrotactile flutters or the perceived coherence of a random dot kinematogram. Therefore, it is expected from this account that the representation in our studies materializes in areas such as the PFC and higher-order parietal regions, which process abstract, supramodal information. More so, this idea may serve to explain the beta band modulation in motor areas before pressing a button we observed in terms of active perceptual memory. In this case, the motor-specific areas would also maintain information over a short period of time relevant to their place in the cortical hierarchy: the motor code for a subsequent button press.
\section{Beta band during decision making}
Beyond a role for working memory, we found an involvement of premotor beta band activity during decision making in a visual task and used previous findings of decision-related beta in tactile tasks to establish an association with concurrent centroparietal signals. Unfortunately, due to technical difficulties with both the vibrotactile stimulation device and the eye-tracking system, an analysis of our MEG data set in relation to decision making was unsuccessful \parencite[cf.][]{Chandler2015}. This is particularly lamentably, because \textcite{Donner2007} used MEG in a RDM task and found that beta band activity predicted the accuracy and not the content of upcoming perceptual reports, for which we found no evidence with EEG. 
The modulation of upper beta band power by subjects’ choices we observed in sequential comparison tasks is in accord with choice signals in monkey LFPs \parencite{Haegens2011} in frequency, timing and cross-species location. More so, the increase in beta power for choices $S2>S1$ vs. $S2<S1$ follows the same direction common to all previous studies \parencite{Haegens2011,Herding2016,Herding2017,Ludwig2018}. Remarkably, when comparing patterns of visual (study 2, figure 4) and tactile studies \parencite[][figure4]{Herding2016} the time-frequency maps and topographies appear incredibly similar even though stimulus processing relied on distinct sensory modalities and the tasks had diverging timings. 

\section{Common ground for visual and tactile sequential comparison tasks}
This similarity between tactile and visual decision making indicates that the underlying process is supramodal and might indeed depend on the motor output rather than the sensory domain as predicted from the intentional framework of decision making \parencite{Shadlen2008}. Further evidence stems from data used in study 3, experiments 1 and 2. Published also as \textcite{Herding2016,Herding2017}, this data demonstrates that depending on the response modality, the choice-selective beta band modulation can be source localized either to the premotor cortex for button press or the FEF for saccade responses. While these tasks used essentially the same sequential frequency comparison, we can now add that also in the comparison of sequentially sampled RDM stimuli the decision-related beta band modulation can be observed. This is particularly interesting, because during motor processing the beta frequency band does not solely represent motor preparation, as historically thought \parencite{Pfurtscheller1981}. Indeed, when participants responded with either their left or right hand, lateralized beta band power over contralateral MI scaled with the process of accumulating evidence for a decision, tracking the evolving decision variable \parencite{Donner2009,OConnell2012}. In our recordings with button-press responses, we used the right index and middle finger, which in addition were counterbalanced across volunteers, eliminating the possibility that contralateral sensorimotor beta band modulations were alone responsible for choice-dependent beta. More so, because in study 2 we used a visual task and found almost identical results, we can separate such decision-beta from the beta oscillations observed during somatosensory perception. If this modulation is indeed independent from perception and is related to the choice information, then we should only be able to find it if a response mapping is provided. Therefore, in study 3, experiments 3+4, also published as \textcite{Ludwig2018}, participants were only provided with the response mapping after a short delay and had to transform the decision information onto a colour code. Interestingly, in those trials where responses could not be immediately transformed into motor commands the beta band was also similarly modulated, but in posterior parietal cortex, not premotor areas. Taken together with our visual and tactile experiments employing direct mappings, this implies that the way we respond determines where in the sensorimotor hierarchy the decision is processed and supports an intentional framework \parencite{Shadlen2008}. Furthermore, our findings indicate that the beta band reflects the categorical, abstract content of a decision, even in the absence of a motor plan. This is particularly interesting, because also broadband centroparietal signals (CPP) have been theorized to reflect the closely related process of accumulating decision evidence.

\section{Common ground for CPP and decision beta}
Both the premotor beta oscillations and the CPP we observed are candidate signals to reflect large-scale neural ensembles expressing the repeated sequential sampling and integration in sensorimotor neurons observed in studies with nonhuman primates \parencite{Gold2007,Hanks2017,Kelly2015,Spitzer2017}. In particular, they correspond well to studies focusing on the role of the PPC for decision formation and its connections with accumulation of motion information from direction-selective neurons in MT, but also with frontal areas such as the FEF \parencite{Ding2012}. Notably, the LIP and FEF can both exhibit ramping up of neural activity and can reach a fixed level prior to saccade responses \parencite{Ding2010,Ding2012,Hanes1996,Roitman2002}. While the function of single neurons in these areas has been studied extensively, the precise role of the large-scale signals recorded with EEG, decision beta and CPP remains uncertain.
The closest link between monkey and our human studies might be found between my second study and \textcite{Wimmer2016}. They recorded LFPs from monkey lateral prefrontal cortex during a sequential comparison task of the speed and direction of two RDM patches. During stimulus perception beta power was reduced, but theta and gamma increased. During working memory beta power encoded the task-relevant $S1$ feature, matching our findings in visual and tactile recordings. After $S2$ onset broadband LFP activity tracked the difference between $S1$ and $S2$ with an early sensory-related component reflecting the stimulus difference and a later component associated with the behavioural decision build-up. This is remarkably similar to our findings concerning the CPP, but in a wholly different area. However, the lPFC is very well-connected with the PPC \parencite{Cole2010} and my own findings add to a well-established relationship between these areas \parencite{Cole2013,Cole2007,Duncan2010,Muhle-Karbe2017,Nieder2016}. These results, in conjunction with ours, indicate that during cognitive tasks a network of prefrontal and parietal areas transition dynamically between neural coding states in a variety of frequency bands rather than one type of oscillation or broadband signal underlying perceptual decision making. The coupling between CPP and choice-related beta in our findings, however, indicates that broadband signals and the beta band fulfil very related roles. I speculate that the CPP may reflect the decision variable in an accumulation to bound manner, while the choice-related beta band serves to communicate the result of this process, continuously over time as is crucial for response preparation. This serves to explain, why choice-related beta appears so early in all our recordings (studies 2 \& 3), about 150 ms after onset of the second stimulus and disappears long before motor action is taken. This observation indicates that choice-related beta is strongest, when the CPP accumulates the most and the most updating of information is necessary \parencite{Twomey2015}. Moreover, this explains also why premotor beta band modulations could only be observed when the response modality was clear \parencite{Ludwig2018}: the information was retained in the PPC till responses could be made. If the choice-related beta was exclusively related to response preparation, we would have observed such a modulation before subjects responded regardless. However, further research into the beta band – CPP relationship will be necessary. 
First, similar to our task design in study 1, it may be interesting to know where beta modulations will originate if there is a delay before responding and the response modality (button press / saccade) is either known or unknown on a trial-by-trial basis. I would hypothesize that in trials with unknown response modality beta band modulations source localize to the PPC, while in trials where subjects know how to answer, this effect originates from premotor areas or FEF. 
Second, it will be important to investigate gamma oscillations in this task, because – as previously discussed – the beta band is theorized to reflect the re-activation of content \parencite{Spitzer2017} and might reflect a maintenance state that is interrupted by short gamma bursts \parencite{Lundqvist2018,Lundqvist2016}. Recent developments based on the Hidden Markov Model \parencite{Vidaurre2018,Vidaurre2016} have been used to identify fast transient states in electrophysiological data, and could be a promising method to characterize such a mechanism. 
Third, we observed a scaling of the CPP with subjectively perceived difficulty and related these changes to statistical decision confidence. A crucial next step will be to record actual ratings of confidence on single trials to uncover how confidence interacts with the accumulation of evidence tracked by the CPP. In particular, it remains unclear, whether at the time of response a fixed threshold bound is reached or if the signal is modulated by confidence \parencite{Gherman2015,Kelly2013,Kelly2015,Philiastides2014,Twomey2016}. For example, our recordings in study 2 demonstrate that the CPP builds-up as expected from an accumulation process but is scaled by trial difficulty at the time of response, which is closely related to confidence. A drift-diffusion model with non-collapsing bounds would have predicted that a fixed threshold of CPP amplitude is reached independent of trial difficulty as observed in similar recordings \parencite{Kelly2013}. For future mechanistic explanations and the large body of modelling work on such decision processes we should identify, if confidence signals, e.g. from the ventral striatum as observed with fMRI \parencite{Hebart2016}, are either mixed with or alternatively directly influence the CPP. 

\section{Outlook: Synaptic codes for dynamic control}
We have observed a common pattern for both working memory and decision making. Historically, neuroscientists have observed direct neural correlates of stimulus properties in firing rates \parencite[e.g.][]{Kim1999,Romo1999} and have subsequently extended this view to populations’ states and large-scale network functions \parencite{Barak2010,Lundqvist2018,Siegel2012}. While this does not preclude important contributions of small populations and single neurons \parencite{Shadlen2013,Shadlen1998}, the question arises whether information maintenance and decision formation rely on small or large sets of neurons. This question goes hand in hand with another: Does the brain use a direct code, as observed for example in V1 orientation columns, or a distributed, as reflected in the large-scale dynamics that give rise to my observations with M/EEG? So far, it appears that both codes exist - but is this computationally plausible? 
Interestingly, the answer may lie in biophysical work on synapses. \textcite{Mongillo2008} propose that working memory is sustained by calcium-mediated synaptic facilitation of cortical networks. This may be necessary, because the sustained firing of many neurons as a regular code would consume too much energy \parencite{Laughlin2001}. Moreover, synaptic excitation can regulate bursting and coherent oscillations on the level of single neurons and networks across very short and long timescales \parencite{Durstewitz2009}. This could explain why neural firing codes are rarely persistent over long periods of time but rather ebb and flow with cognitive demands in a re-entry of information function \parencite{Compte2000,Fuster2012,Fuster2015,Stokes2015,Verduzco-Flores2009,Wang2013}. However, to test such a mechanism experimentally one would need simultaneous recordings from neurons with direct monosynaptic connections, whose firing is correlated. Lamentably, the chance of selecting neurons with such connections is very low (1-2\%), as evidenced by rat PFC recordings during a maze task giving first evidence for synaptic WM models \parencite{Fujisawa2008}. Notably, such findings during working memory are likely to extend to other cognitive functions and could be particularly important for WM-based decision making as studied in this thesis \parencite[see also][]{Stokes2013}. In sum, while direct evidence for synaptic codes are still to be studied in humans, this idea provides a promising avenue of research uniting biophysical descriptions and cognitive theories of WM and decision making.

\section{General Summary}
As climate change leads to ever higher temperatures \parencite{Parmesan2003}, buying good quality watermelons will become more important. So, what can we say about the neural basis of finding the right melon? First, when we feel the watermelon, the power of neural oscillations in the alpha band increase over task-irrelevant visual areas. Concurrently, low beta (mu) oscillations increase over the ipsilateral hemisphere of the hand doing the feeling, while decreasing over the contralateral. Then, while we keep how the watermelon felt in memory before selecting another to test, beta band power increases with this abstract quantity, while gamma decreases. Finally, before we point to the watermelon we want to buy, broadband centroparietal signals characterize the process of accumulating evidence for one melon or the other while beta band activity from premotor areas reflects our choice. These large-scale neural oscillations reflect the dynamic, fast-paced changes in single neurons and neuronal populations that unite their rhythms, making them detectable with neuroimaging methods from outside the human skull. 
