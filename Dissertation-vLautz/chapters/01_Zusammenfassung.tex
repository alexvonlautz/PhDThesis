% !TEX root = ../Dissertation-vLautz.tex

\chapter*{Zusammenfassung}
\doublespacing
Der Mensch trifft täglich tausende Entscheidungen, von komplexen ("`Gehe ich über die rote Ampel?"'), über abstrakte ("`Buchstabiere ich Foto mit F oder Ph?"'), zu sensorisch geprägten ("`Hat mein Telefon gerade vibriert?"'). Das Gehirn muss dabei flexibel auf eine Vielzahl von sensorischen Reizen und Entscheidungstypen reagieren. Um dies zu ermöglichen, arbeiten viele Neurone in ganz unterschiedlichen kortikalen Arealen zusammen. Neurowissenschaftler vermuten, dass kortikale Oszillationen dabei eine zentrale Rolle spielen. Sie reflektieren das Zusammenwirken vieler Neurone und werden zur Kommunikation neuronaler Populationen genutzt. In der vorliegenden Arbeit wurde die Rolle einzelner Rhythmen für das kurzfristige Speichern von Informationen im Arbeitsgedächtnis, sowie das Treffen von Entscheidungen untersucht.

Dieser Dissertation liegen drei Studien zugrunde, im Rahmen derer Versuchspersonen zwei nacheinander dargebotene Stimuli vergleichen sollten. Um diese Aufgabe zu bewältigen, muss der erste Stimulus kurz im Gedächtnis behalten werden. Dann kommt es zur eigentlichen Entscheidung, dem Abgleich der  beiden Stimuli. Bei diesem Versuch konnten wir mit Hilfe von Magneto- und Elektroenzephalographie Oszillationen messen, die sich mit den Stimuluseigenschaften veränderten. Arbeitsgedächtnisprozesse waren dabei mit Beta-Oszillationen assoziiert. Insbesondere zeigten sich parietale und präfrontale Beta-Oszillationen, die mit Gamma-Oszillationen im präfrontalen Kortex zusammenspielten. Daher vermuten wir, dass ein fronto-parietales Netzwerk für das Behalten von Stimulus-Information von Bedeutung ist und diskutieren im Folgenden zugrundeliegende Mechanismen. Außerdem konnten wir während des Treffens von Entscheidungen prämotorische Beta-Oszillationen messen, die 0,7 Sekunden vor der Antwort der Versuchsteilnehmer die Entscheidung reflektierten. Darüber hinaus zeigten sich auch bekannte parietale Signale, die den Prozess der Entscheidungsfindung abbildeten. Interessanterweise waren diese Signale vom Schwierigkeitsgrad der Aufgabe abhängig. Dies spricht dafür, dass aktuelle Theorien zu neuronalen Grundlagen der Entscheidungsfindung ergänzt werden müssen.