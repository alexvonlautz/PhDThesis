% !TEX root = ../Dissertation-vLautz.tex

\chapter*{Zusammenfassung}
\doublespacing
Der Mensch trifft täglich tausende Entscheidungen, von komplexen „Gehe ich über die rote Ampel?“, über abstrakte „Buchstabiere ich Foto mit F oder Ph?“, zu sensorisch geprägten „Hat mein Telefon gerade vibriert?“. Das Gehirn muss dabei flexibel auf eine Vielzahl von sensorischen Reizen und Entscheidungstypen reagieren können. Um diese Funktionen zu ermöglichen, arbeiten viele Neurone in ganz unterschiedlichen kortikalen Arealen zusammen. Wie aber kommunizieren und verständigen ich Neurone auf eine Weise, die flexible und effektive Reaktionen ermöglicht? Neurowissenschaftler vermuten, dass dabei kortikale Oszillationen eine zentrale Rolle einnehmen. Sie reflektieren das Zusammenwirken vieler Neurone und werden zur Kommunikation neuronaler Populationen genutzt. In dieser Dissertation habe ich in drei Studien die Rolle einzelner oszillatorischer Rhythmen für kurzfristiges Speichern von Informationen im Arbeitsgedächtnis, sowie das Treffen von Entscheidungen untersucht. In diesen Experimenten erhielten Versuchspersonen eine einfache Aufgabe, nämlich zwei nacheinander dargebotene Stimuli zu vergleichen. Diese Aufgabe ist gut geeignet, umsequenzielle mentale Schritte zu untersuchen. Zunächst muss der erste Stimulus wahrgenommen werden, anschließend muss dieser gespeichert werden, bis der zweite dargeboten wird. Nachdem der zweite wahrgenommen wurde, muss die Erinnerung des ersten mit dem zweiten abgeglichen werden, um abschließend eine Entscheidung zu treffen.

In der ersten Studie verglichen die Probanden zwei vibrotaktile Frequenzen während Magnetoenzephalographie (MEG) Daten aufgezeichnet wurden. MEG wurde bisher noch nicht zur Untersuchung dieser Arbeitsgedächtnis-Aufgabe benutzt und ist im Vergleich zur Elektroenzephalographie (EEG) besser zur Detektion hoher neuronaler Frequenzen und der genaueren Bestimmung von kortikalen Quellen geeignet. Tatsächlich konnte ich in dieser Studie eine parametrische Modulation der hochfrequenten Gamma Oszillationen im präfrontalen Kortex während der kurzfristigen Speicherung von taktiler Information feststellen. Zudem fand ich eine monotone Enkodierung der vibrotaktilen Stimulusfrequenz im Beta Frequenzband, sowohl im präfrontalen als auch im parietalen Kortex. Im zweiten Experiment nutzten wir den Vergleich zweier visueller Stimuli und konnten mit Hilfe von EEG ebenfalls eine Rolle des Beta Frequenzbandes für die kurzzeitige Speicherung von visueller Information zeigen.  Zudem enkodierte das Beta Frequenzband in prämotorischen Arealen die Entscheidung der Versuchspersonen schon 0.7s bevor sie diese kommunizierten. Dieses Ergebnis ist ein Hinweis darauf, dass Entscheidungen nicht in einem zentralen Entscheidungsareal getroffen werden, sondern in Abhängigkeit davon, welche Konsequenzen die Entscheidung hat. Während des Entscheidungsprozesses war zusätzlich das bekannte parietale Signal, die P300, erkennbar. Diese war von der Entscheidung beeinflusst und ihre maximale Ausprägung zum Zeitpunkt der Antwort war von der Schwierigkeit jeden Durchlaufs abhängig. In der dritten Studie untersuchten wir diesen Effekt in sechs Experimenten genauer und konnten einen Zusammenhang der P300 mit der Sicherheit in der Entscheidung und mit der Modulation des Beta Frequenzbandes zeigen. Kortikale Oszillationen spielen daher beim Treffen einer ganzen Reihe an Entscheidungen eine wichtige Rolle und insbesondere das Beta Frequenzband scheint bei der Speicherung und Aufbereitung von Information zum Treffen von Entscheidungen eine besondere Funktion zu erfüllen.

