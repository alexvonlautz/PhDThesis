% !TEX root = ../Dissertation-vLautz.tex

\chapter*{Abstract}
In our daily lives we are faced with thousands of decisions: from complex ‘should I cross while it's on a red light?’, to abstract ‘do I spell color with o or ou?’, to sensory dominated questions like ‘did my phone just vibrate?’. To navigate all of these different types of decisions, the brain has to incorporate a plethora of information from sensory and memory systems, requiring many neuronal populations from distinct cortical areas to work together. Neuroscientists posit that cortical oscillations play an important part in this process. I investigated the role of such cortical rhythms for the short retention of information in working memory and decision making with three experimental studies.

In all experiments, participants were asked to compare two sequentially presented stimuli. To solve this task, the first stimulus has to be kept in memory for a short while and is then compared to the second. While participants held the first stimulus in memory, magneto- and electroencephalographic recordings revealed  a parametric modulation of parietal and prefrontal beta oscillations with the to-be-remembered stimulus feature. At the same time, we observed a previously unknown prefrontal gamma power decrease that was negatively correlated with the beta band effects. Therefore we suspect that there is a fronto-parietal network that communicates in these two frequency bands during working memory. In addition, we found decision-related activity in premotor beta power that encoded participants’ choices 0.7 seconds before they enacted their responses. Moreover, we also found a well-known parietal signal, which tracked the evolution of the decision over time. Interestingly, this signal was modulated by the difficulty of the decisions, indicating that present theories about perceptual decision making need to be extended.