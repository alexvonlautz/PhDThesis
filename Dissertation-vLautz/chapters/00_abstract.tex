% !TEX root = ../Dissertation-vLautz.tex

\chapter*{Abstract}
In our daily lives we are faced with thousands of decisions: from complex ‘should I cross the red light’, over abstract ‘do I spell color with o or ou?’, to sensory dominated questions like ‘did my phone just vibrate?’. To navigate all of these different types of decisions, the brain has to incorporate a plethora of information from sensory and memory systems, requiring many neuronal populations from distinct cortical areas to work together. But how do they communicate and understand each other to enable flexible control when faced with a multitude of possible decisions? Neuroscientists posit that cortical oscillations play an important part in this process. They reflect the activity of many neurons in concert and are used to communicate across distances between neuronal populations. In this dissertation I investigated the role of such cortical rhythms for the short retention of information in working memory and decision making with three experimental studies. In all experiments, participants were given a simple task. They were asked to compare two sequentially presented stimuli. This task is well-suited to separate distinct mental steps necessary to perform the comparison. First, the first stimulus has to be perceived. Second, the first stimulus has to be kept in memory for a short while. Third, after perception of the second stimulus, the memory of the first has to be compared with the second stimulus. 

In the first study I let participants compare two vibrotactile frequencies delivered to the index finger and measured magnetoencephalography (MEG), which in comparison with electroencephalogaphy (EEG) serves better to investigate high frequency oscillations and enables more accurate source localization. Indeed, we were able to find a previously unknown parametric modulation of high frequency gamma oscillations in prefrontal areas during the short retention of tactile information. In addition, we found a monotone encoding of the stimulus frequency in the beta band, both in prefrontal and parietal cortices. The second experiment used EEG and the comparison of two visual stimuli to establish a supramodal role of beta oscillations for parametric working memory. In addition, beta power in premotor areas also encoded participants’ choices 0.7 seconds before responding. This indicates that the decision is not made in one central brain area, but in those related to the decision outcome. During the decision process, we also found a well-known parietal signal, the P300, which tracked the evolution of the decision and whose peak amplitude at the time of response was modulated by trial difficulty. The third study investigated this effect in six experiments more thoroughly and found a relationship of the P300 with decision confidence and decision-related beta oscillations. Thus, we found that cortical rhythms play an important role during information retention and decision formation and suggest an extended, content-specific role for the beta frequency band in this process.

