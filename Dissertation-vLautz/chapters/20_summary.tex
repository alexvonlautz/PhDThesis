% !TEX root = ../Dissertation-vonLautz.tex

\chapter{Summary of Original Research Articles}

%%%%%%%%%%%%%%%%%%%%%%%%%%%%%%%%%%%%%%%%%%%%%%%%%%

\section{Study 1: Gamma and Beta Oscillations in Human MEG Encode the Contents of Vibrotactile Working Memory}
From single-cell recordings in monkeys to large-scale human EEG, the parametric encoding of vibrotactile frequency during working memory is well-established \parencite{Romo1999,Spitzer2010}. As reviewed in the introduction, a number of recent EEG studies have identified cortical oscillations in the beta band to represent the frequency information during a short delay in a vibrotactile sequential frequency comparison task. However, visual and auditory working memory studies have found a crucial role of gamma oscillations for working memory, not observed in previous vibrotactile EEG studies \parencite{Roux2014}. In addition, the only MEG study investigating tactile working memory found a modulation of SI and - most interestingly - of SII during stimulus retention, but didn’t investigate a parametric modulation in prefrontal areas \parencite{Haegens2010}. While the authors did find an overall increase in frontal gamma during WM, this was limited to contrasting periods of working memory with a prestimulus baseline and was not content specific. Therefore, our first goal was to investigate whether frontal gamma encodes stimulus features during WM. In addition to this aim, a line of monkey and human neuroimaging studies has identified the intraparietal sulcus (IPS) as a hub for numerosity processing, a mental task very similar to the vibrotactile features in our design \parencite{Nieder2016}. However, studies using EEG have not been able to localize IPS activity reflecting the vibrotactile frequency held in WM as is known from fMRI \parencite{Wu2018}. Therefore, we aimed at detecting both high frequency oscillations in the gamma band and stimulus information from the IPS for the first time.

We recorded 306-channel whole-head magnetoencephalography while participants performed a version of the vibrotactile frequency comparison task. Notably, our original task design intended to also investigate how beta-gamma codes of decision making were influenced by foreknowledge of the response modality, and therefore the task included responses with button press and saccades. However, due to technical problems, we had to discard the analysis of decision making related activity. 

Our main analysis probed with a zero-mean contrast of the four different vibrotactile frequencies held in working memory at 15, 19, 23 and 27 Hz, whether there were channels, frequencies or time points in which a parametric code of vibrotactile frequencies was present. With a nonparametric cluster-based permutation test we identified three areas that showed such a pattern, all around the center of the WM interval. First, replicating previous EEG findings, beta power in the right IFG at around 30-35 Hz. Second, low beta power (10-20Hz) in bilateral parietal channels which was source localized to the IPS. Third, matching the beta effect in source location in the right IFG, gamma power (74-90 Hz) was negatively modulated by the vibrotactile frequency, thus showing the opposite pattern of the beta band effects. Notably, we did not replicate effects of overall broadband gamma power increases in SI, SII and frontal cortices as had been observed in a similar vibrotactile task \parencite{Haegens2010}, while replicating the patterns typically observed in vibrotactile SFC tasks with M/EEG \parencite{Bauer2006,Spitzer2010}. 

These results indicate that there is a frontoparietal network underlying the retention of vibrotactile stimuli with an extended role of the beta band that may interact with gamma to enable working memory. We demonstrate for the first time with MEG that the IPS is also involved in this process and that the gamma band, which is associated more directly with neuronal firing than beta, might drive the prefrontal processing \parencite{Lundqvist2016,Whittingstall2009}. 

\section{Study 2: Centro-parietal EEG Potentials Index Subjective Evidence and Difficulty}

Recent studies indicate that the CPP in human EEG  tracks the integration of noisy sensory input over time. It remains unclear however, whether the comparison of two short vibrotactile frequencies elicits a similar response, as this percept does not include the accumulation of noisy evidence over an extended period. Moreover, it remains unknown whether the premotor beta band that scales with subjects’ upcoming choices is related to the CPP. This is particularly interesting, because the CPP reflects the absolute, unsigned strength of accumulated evidence ($|S2-S1|$), while choice dependent beta is a signed value ($S2-S1$) reflecting the direction of the decision \parencite[e.g., ][]{Siegel2011,Urai2014}. 

In this study, we used EEG data from six variants of the SFC task (n=116) and applied a model based on Bayesian inference to the behavioural data to estimate the subjectively perceived frequency differences (SPFDs). We found that parietal ERPs reflected the SPFCs shortly after the second stimulus offset (168-709ms) in a signed fashion, thus indicating the direction of choice. Crucially, this early parietal signal was correlated with choice-related beta power on a single trial level. While not implying causation, this can be viewed as first evidence for a previously unknown EEG signature that indexes the updating of subjective evidence in relation to the ensuing choice. The timing of these two signals as well as their source locations in parietal and premotor cortex further underline the possibility that the early ERPs serve to communicate the evidence for one motor plan or its alternative and the premotor beta band reflects the choice planning based upon this evidence.

In addition to this early parietal modulation by signed difference, we observed later parietal ERPs (273-953ms after stimulus offset) that correlated with the absolute strength of evidence. Interestingly, this later modulation was source localized not only to parietal areas, but also included the bilateral IFG, which relates it to the parametric encoding of the vibrotactile frequencies during WM. Similar to study 3 we also did not observe an absolute threshold of CPP accumulation, but rather a scaling of the late ERPs by subjective task difficulty. Further analysis revealed that this effect complied with the definition of statistical decision confidence in all aspects \parencite{Hangya2016,Sanders2016}. 

In summary, these findings indicate that early centroparietal ERPs reflect the evidence on which decisions are based, while later modulations might refer to the strength of evidence informing a decision, which is closely related to measures of confidence.

\section{Study 3: Neuronal Signatures of a Random-Dot Motion Comparison Task}
Both the vibrotactile frequency comparison task and the random-dot motion task have been studied extensively in humans and monkeys, as I introduced earlier. However, there is remarkably little research aiming at finding common codes across these perceptual decision making tasks. So far, no human neuroimaging study has investigated the RDM task with vibrotactile stimuli nor studied the sequential comparison of random-dot motion stimuli. This is particularly curious, because pupillometry in humans and electrophysiological recordings in monkeys using a combination of these classical tasks have produced high-impact studies that gave novel insights into the encoding of decision information \parencite{Urai2017,Wimmer2016}. Here, we recorded EEG while human volunteers were tasked to compare the coherence of two sequentially presented random-dot motion stimuli ($S1$, $S2$) and responded by button press. If findings from SFC studies indeed transfer to the visual domain, we should observe a parametric beta band code in PFC as well as a modulation of premotor beta by choice. Moreover, RDM stimuli should elicit typical visual effects as well as a correlate of the accumulation of evidence in the form of a CPP. Crucially, the decision variable in this task reflects the comparison of the first with the second stimulus, thus we should not observe a CPP during perception of $S1$. Moreover, the CPP should scale with the difference between the two, not the coherence of $S2$.

We asked 28 subjects to perform this task and used their behavioral data to model the subjectively perceived coherence difference (SPCD) for each subject and trial. Using variational Bayes, our model accounted for the time-order effect/error \parencite{Hellstrom1985}, a bias typically observed in sequential comparisons. In analogy to study 1, our WM analysis was a parametric contrast of the four $S1$ coherence levels. To look into the decision making interval, we performed a $2x2$ GLM of choice ($S2>S1$ vs $S2<S1$) and performance (correct vs incorrect). Moreover, we investigated the CPP and contrasted it by subjects’ choices and each trial’s SPCD in three levels of ‘easy’, ‘medium’ and ‘hard’.

In our WM analysis we found a significant cluster of prefrontal channels that were modulated by the level of $S1$ coherence retained throughout the inter-stimulus-interval. In agreement with previous EEG studies using vibrotactile stimuli \parencite{Spitzer2010,Spitzer2012}, this effect source localized to the right IFG, suggesting that the working memory related beta is supramodal and can be observed with stimuli not relying on frequency magnitudes. We also found a negative modulation of gamma, replicating findings from study 1. However, this effect was mainly driven by the lowest coherence stimuli and will require further investigation. Curiously, we also found a negative modulation of low beta band activity by the $S1$ coherence level in centroparietal channels, source localizing to bilateral MI and precuneus.

The analysis of decision-related activity found a modulation of premotor beta band activity 700ms before responses were made that was elevated for choices of $S2>S1$ in comparison to those of $S2<S1$. This is in line with recent vibrotactile SFC studies in humans, corresponding in time, frequency and location \parencite{Herding2016} and also agrees with monkey LFPs \parencite{Haegens2011,Haegens2017}. There was no effect when splitting trials into 3 levels by SPCD (easy, medium, hard), thus reflecting the choices in a binary code. 
The CPP was modulated both in response to the perception of $S2$ and with respect to responses. $S2$-locked activity accrued during the stimulus presentation and stayed on a fixed level afterwards. Crucially, the amplitude of this level was modulated by the trials’ difficulty ($S2-S1’$) and not by the coherence of $S2$. Furthermore, the $S2$-locked CPP was modulated by choice and reached a higher amplitude for choices $S2<S1$ than $S2>S1$, the opposite of the beta band effect. Response-locked CPP showed a pattern of signal accumulation to a peak at the time of response. Notably, this peak was both influenced by choice and the difficulty of trials. At the time of response the CPP did not reach a fixed threshold, like in simple boundary-crossing models, but was scaled by the SPCD, with difficult trials exhibiting smaller amplitudes and incorrect trials demonstrating even smaller amplitudes. This effect was only evident in the last 300ms before responding and in particular, seemed to be driven by a lower starting point to the accumulation rather than variance during the accumulation.

Our findings indicate an extended role of the beta band for both working memory and decision making in comparison tasks, regardless of sensory modality. Beta has been suggested to reflect the “status quo” of information \parencite{Engel2010}. In our studies however, it appears to reflect more than that. It is modulated by the abstract magnitudes in comparison tasks (vibrotactile frequency or RDM coherence) and therefore reflects the WM content, as well as the content of decision making, already very early, 700ms before responding. In conjunction with fast, transient gamma it might therefore reflect the re-activation of content \parencite{Spitzer2017} and/or the maintenance state (also of choice) that is interrupted by gamma \parencite{Lundqvist2018}.

The CPP was strongly modulated by trial difficulty at the time of response, suggesting it reflects a cognitive process that is not wholly explained by crossing a bound in a simple drift-diffusion model. More complex models based on sequential Bayesian updating or with collapsing bounds may be necessary to keep the drift-diffusion view in place. Moreover, it is possible that the CPP reflects an accumulation to an absolute bound, but the signal we observed included other parietal signals encoding the decision confidence. Finally, the CPP was also modulated by participants' choices, indicating a relationship with premotor beta that is yet to be investigated.

In sum, this study was able to bridge gap between decision making paradigms and sensory domains, indicating a common role for beta band driven content encoding and the CPP as an evidence accumulation mechanism that has ties to confidence.
